\batchmode
\documentclass{howto}
\makeatletter


\title{Regular Expression HOWTO}


\release{0.00}


\author{A.M. Kuchling}
\authoraddress{\email{akuchling@acm.org}}



\usepackage[dvips]{color}
\pagecolor[gray]{.7}



\makeatletter

\makeatletter
\count@=\the\catcode`\_ \catcode`\_=8 
\newenvironment{tex2html_wrap}{}{} \catcode`\_=\count@
\makeatother
\let\mathon=$
\let\mathoff=$
\ifx\AtBeginDocument\undefined \newcommand{\AtBeginDocument}[1]{}\fi
\newbox\sizebox
\setlength{\hoffset}{0pt}\setlength{\voffset}{0pt}
\addtolength{\textheight}{\footskip}\setlength{\footskip}{0pt}
\addtolength{\textheight}{\topmargin}\setlength{\topmargin}{0pt}
\addtolength{\textheight}{\headheight}\setlength{\headheight}{0pt}
\addtolength{\textheight}{\headsep}\setlength{\headsep}{0pt}
\setlength{\textwidth}{349pt}
\newwrite\lthtmlwrite
\makeatletter
\let\realnormalsize=\normalsize
\global\topskip=2sp
\def\preveqno{}\let\real@float=\@float \let\realend@float=\end@float
\def\@float{\let\@savefreelist\@freelist\real@float}
\def\end@float{\realend@float\global\let\@freelist\@savefreelist}
\let\real@dbflt=\@dbflt \let\end@dblfloat=\end@float
\let\@largefloatcheck=\relax
\def\@dbflt{\let\@savefreelist\@freelist\real@dbflt}
\def\adjustnormalsize{\def\normalsize{\mathsurround=0pt \realnormalsize
 \parindent=0pt\abovedisplayskip=0pt\belowdisplayskip=0pt}\normalsize}%
\def\lthtmltypeout#1{{\let\protect\string\immediate\write\lthtmlwrite{#1}}}%
\newcommand\lthtmlhboxmathA{\adjustnormalsize\setbox\sizebox=\hbox\bgroup\kern.05em }%
\newcommand\lthtmlvboxmathA{\adjustnormalsize\setbox\sizebox=\vbox\bgroup\kern.05em %
 \let\ifinner=\iffalse }%
\newcommand\lthtmlboxmathZ{\@next\next\@currlist{}{\def\next{\voidb@x}}%
 \expandafter\box\next\egroup}%
\newcommand\lthtmlmathtype[1]{\def\lthtmlmathenv{#1}}%
\newcommand\lthtmllogmath{\lthtmltypeout{l2hSize %
:\lthtmlmathenv:\the\ht\sizebox::\the\dp\sizebox::\the\wd\sizebox.\preveqno}}%
\newcommand\lthtmlfigureA[1]{\let\@savefreelist\@freelist
       \lthtmlmathtype{#1}\lthtmlvboxmathA}%
\newcommand\lthtmlfigureZ{\lthtmlboxmathZ\lthtmllogmath\copy\sizebox
       \global\let\@freelist\@savefreelist}%
\newcommand\lthtmldisplayA[1]{\lthtmlmathtype{#1}\lthtmlvboxmathA}%
\newcommand\lthtmldisplayB[1]{\edef\preveqno{(\theequation)}%
  \lthtmldisplayA{#1}\let\@eqnnum\relax}%
\newcommand\lthtmldisplayZ{\lthtmlboxmathZ\lthtmllogmath\lthtmlsetmath}%
\newcommand\lthtmlinlinemathA[1]{\lthtmlmathtype{#1}\lthtmlhboxmathA
  \vrule height1.5ex width0pt }%
\newcommand\lthtmlinlineA[1]{\lthtmlmathtype{#1}\lthtmlhboxmathA}%
\newcommand\lthtmlinlineZ{\egroup\expandafter\ifdim\dp\sizebox>0pt %
  \expandafter\centerinlinemath\fi\lthtmllogmath\lthtmlsetinline}
\newcommand\lthtmlinlinemathZ{\egroup\expandafter\ifdim\dp\sizebox>0pt %
  \expandafter\centerinlinemath\fi\lthtmllogmath\lthtmlsetmath}
\newcommand\lthtmlindisplaymathZ{\egroup %
  \centerinlinemath\lthtmllogmath\lthtmlsetmath}
\def\lthtmlsetinline{\hbox{\vrule width.1em\kern-.05em\vtop{\vbox{%
  \kern.1em\copy\sizebox}\ifdim\dp\sizebox>0pt\kern.1em\else\kern.3pt\fi
  \ifdim\hsize>\wd\sizebox \hrule depth1pt\fi}}}
\def\lthtmlsetmath{\hbox{\vrule width.1em\kern-.05em\vtop{\vbox{%
  \kern.1em\kern0.8 pt\hbox{\hglue.17em\copy\sizebox\hglue0.8 pt}}\kern.3pt%
  \ifdim\dp\sizebox>0pt\kern.1em\fi \kern0.8 pt%
  \ifdim\hsize>\wd\sizebox \hrule depth1pt\fi}}}
\def\centerinlinemath{%
  \dimen1=\ifdim\ht\sizebox<\dp\sizebox \dp\sizebox\else\ht\sizebox\fi
  \advance\dimen1by.5pt \vrule width0pt height\dimen1 depth\dimen1 
 \dp\sizebox=\dimen1\ht\sizebox=\dimen1\relax}

\def\lthtmlcheckvsize{\ifdim\ht\sizebox<\vsize 
  \ifdim\wd\sizebox<\hsize\expandafter\hfill\fi \expandafter\vfill
  \else\expandafter\vss\fi}%
\makeatletter \tracingstats = 1 


\begin{document}
\pagestyle{empty}\thispagestyle{empty}%
\lthtmltypeout{latex2htmlLength hsize=\the\hsize}%
\lthtmltypeout{latex2htmlLength vsize=\the\vsize}%
\lthtmltypeout{latex2htmlLength hoffset=\the\hoffset}%
\lthtmltypeout{latex2htmlLength voffset=\the\voffset}%
\lthtmltypeout{latex2htmlLength topmargin=\the\topmargin}%
\lthtmltypeout{latex2htmlLength topskip=\the\topskip}%
\lthtmltypeout{latex2htmlLength headheight=\the\headheight}%
\lthtmltypeout{latex2htmlLength headsep=\the\headsep}%
\lthtmltypeout{latex2htmlLength parskip=\the\parskip}%
\lthtmltypeout{latex2htmlLength oddsidemargin=\the\oddsidemargin}%
\makeatletter
\if@twoside\lthtmltypeout{latex2htmlLength evensidemargin=\the\evensidemargin}%
\else\lthtmltypeout{latex2htmlLength evensidemargin=\the\oddsidemargin}\fi%
\makeatother

% !!! IMAGES START HERE !!!

\stepcounter{section}
\stepcounter{section}
\stepcounter{subsection}
\stepcounter{subsection}
\stepcounter{section}
\stepcounter{subsection}
\stepcounter{subsection}
\stepcounter{subsection}
\stepcounter{subsection}
\stepcounter{subsection}
\stepcounter{section}
\stepcounter{subsection}
\stepcounter{subsection}
\stepcounter{subsection}
\stepcounter{section}
\stepcounter{subsection}
{\newpage\clearpage
\lthtmlfigureA{method468}%
\begin{method}{split}{string, \optional{, maxsplit\code{ = 0}}}
  Split \var{string} by the occurrences of \var{pattern}.  If
  capturing parentheses are used in pattern, then occurrences of
  patterns or subpatterns are also returned.
  If \var{maxsplit} is nonzero, at most \var{maxsplit} splits
  are performed.
\end{method}%
\lthtmlfigureZ
\lthtmlcheckvsize\clearpage}

\stepcounter{subsection}
{\newpage\clearpage
\lthtmlfigureA{method488}%
\begin{method}{sub}{replacement, string\optional{, count\code{ = 0}}}
Returns the string obtained by replacing the leftmost non-overlapping
occurrences of the pattern in \var{string} by the replacement
\var{replacement}.  If the pattern isn't found, \var{string} is returned
unchanged.  
\par
The optional argument \var{count} is the maximum number of pattern
occurrences to be replaced; \var{count} must be a non-negative
integer.  The default value of 0 means to replace all occurrences.
\end{method}%
\lthtmlfigureZ
\lthtmlcheckvsize\clearpage}

\stepcounter{section}
\stepcounter{subsection}
\stepcounter{subsection}
\stepcounter{subsection}
\stepcounter{subsection}
\stepcounter{section}

\end{document}
